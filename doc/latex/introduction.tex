\chapter{Introduction\label{cha:Introduction} }

This manual describes the results of nearly 20 years of effort (much
of it the ``on again, off again'' variety). ModAEM has been through
four major revisions, several research branches, and has been used
for countless wellhead protection models, even though most ModAEM
users never knew they were using it! This chapter is to introduces
to the history and philosophy of ModAEM. 


\subsection*{Early ModAEM}

The essential design of ModAEM was conceived during a series of long
discussions between Mark Bakker, Steve Kraemer, and myself during
a High Performance Computing Consortium workshop at EPA's Research
Triangle Park facility in 1995 (I was first introduced to the ``new''
GMS preprocessor at the same meeting). I'm not sure that Steve and
Mark remember the meeting or the design ideas, but they basically
remain intact:
\begin{itemize}
\item Logical separation of the analytic element \emph{functions} (dipoles/doublets,
wells, etc.) from analytic element \emph{applications} (wells, rivers,
inhomogeneities)
\item Implementation of efficient functions for a few very flexible analytic
element functions (at that time, only wells, dipoles of complex strength,
and ponds were anticipated)
\item Make it all parallelizable with very little effort
\item Abandon FORTRAN 77 in favor of a more modern programming language
\end{itemize}
I have been asked many times why ModAEM was written in Fortran-90
(and now in Fortran-95), rather than in a ``better'' language like
C++. I struggled with the choice, but in the end I was advised by
the folks at CICA (Center for Innovative Computer Applications) that
Fortran 90 and High-Performance Fortran was a more flexible and workable
alternative for parallel hardware. On many occasions, I've wished
I had a fully object--oriented language like C++, but I've never wished
I'd had all the migration issues as C++ has evolved for the past decade.
Fortran 90/95 was a fine choice. Now, I can compile ModAEM with OpenMP
for multi-core processors using a totally free toolchain!


\subsection*{WhAEM for Windows}

Later in 1995, Henk Haitjema and I were hired by US EPA to develop
a new version of the WhAEM for DOS code that had been released in
1994. The new WhAEM was to run under Windows 95 and its successors.
We needed a free solver engine that had available source code (GFLOW
was fully proprietary back then), so we decided that I should complete
a simple ModAEM for use in WhAEM. This code eventually became ModAEM-1.0,
and it was the computational heart of WhAEM until it was replaced
with a stripped--down GFLOW in 2002.

ModAEM-1.0, or ``EPA ModAEM'' is a public--domain code that supports
discharge--specified wells, discharge--specified and head--specified
line sinks, no--flow boundaries, uniform flow, and recharge ponds.
It also does streamline tracing and generates grids for contour lines.
To the best of my knowledge, EPA still has the source code on the
CoSMOS web site.

In addition to its use in WhAEM, I used ModAEM-1.0 as a research code
for local 3-D models in my disseratation work. Very little of that
code exists now.

\subsection*{The SFWMD years}

In 1998, I finished my Ph.D. at Indiana and began working for the
South Florida Water Management District in West Palm Beach, FL. I
did some work with ModAEM, typically in the guise of WhAEM, and implemented
a few additional features, including resistance line sinks. I also
did experimental work on a regional transient flow model based on
ModAEM. ModAEM was a useful research project, but there were few new
applications. 

In 2000, my wife, infant son, and I returned to Bloomington, where
I joined WHPA Inc., a small consulting firm led by my best friend,
Jack Wittman.

\subsection*{ModAEM reborn -- the Idaho delineation project}

In late 2000, WHPA teamed with Barr Engineering on a challenging project:
the delineation of wellhead protection areas for over 450 wells in
the Treasure Valley, near Boise. Two previous MODFLOW models, based
on grid cells of a half--mile or more in size, were available. We
needed a way to quickly delineate accurate capture zones for the wells
based on the MODFLOW flow fields. I proposed that we rebuild ModAEM
for the purpose, using the integrated--flux boundary conditions to
chop sub--domains out of the MODFLOW models; we could then trace particles
in the equivalent analytic element domains. 

This effort required some substantial enhancements to ModAEM: inhomogeneities
in transmissivity, rectangular area sinks, and bounded model domains.
In addition, I restructured the internals of ModAEM to make it more
object--based, including an ``iterator'' strategy that improved
the independence of the code components.

The resulting code became ModAEM-1.2. I planned to make a full release
of ModAEM-1.2 when the manual was ready, but time was limited, and
it was not widely distributed. For some internal applications, I added
head--specified wells, general polygonal area sinks and 3--D pathline
tracing. I began looking for other developers who would be interested
in developing ModAEM, and decided to release a version under the GNU
General Public License, as soon as I could. 

And by the way, we finished the job on time and within the budget!


\subsection*{ModAEM in GMS?}

After the completion of the Idaho project, Jack and I presented the
design and philosphy behind ModAEM in a seminar at EPA's Athens Lab.
The positive feedback I received from the software engineers there
was encouraging. Shortly thereafter, Norm Jones, the inventor of GMS,
contacted me to inquire about the potential for using ModAEM as an
analytic element solver for GMS. Norm and I had numerous discussions,
and a plan was developed for the work. I set out to add some functions
that would make the model sufficiently generic to be useful in GMS:
base and thickness inhomogeneities, including common boundaries; improved
streamline tracing; improved bounded domains with general--head boundaries;
drains; and a number of GUI-related improvements were needed.

Plus, a manual. An early version of this manual was provided to Alan
Lemon at BYU, and he built the GMS front--end for ModAEM. That work
led to the release of ModAEM-1.4 in February, 2004.

\subsection*{What's Next?}

After many years of work and several false starts, ModAEM is now a
versatile modeling tool with a growing community of users. I am hopeful
that a developer community will develop, and ModAEM will continue
to grow in flexibility, performance, and modeling power. If you're
interested in any aspect of ModAEM development, testing, validation,
documentation, or if you want to fund the effort somehow, please visit
the ModAEM Home Page at \texttt{http://www.wittmanhydro.com/modaem.}

Like me, ModAEM now has two children. Mark Bakker and I worked together
on the design of the fully object--oriented Python AE code Tim (now
TimSL). Nearly all of the solver logic and the internal organization
is derived from ModAEM's solver; however since Tim is a research code,
the parallelism--improving separation of functions from applications
is omitted. TimSL now has a sibling, TimML, Mark Bakker's Bessel--function
based model. 

The Tim project provides a set of tools for analytic element education
and research. ModAEM remains a stable, high--performance production
code. Both projects continue to cross--pollinate each other. I expect
that I will continue to develop and model with both codes for a long
time to come. 


\section{The Philosophy of ModAEM}

ModAEM has a set of governing principles. These have been constant
throughout the development of the code and I don't anticipate changing
them:
\begin{description}
\item [{Keep~it~free}] This is embodied in the choice of the GNU General
Public License for ModAEM-1.4 and beyond. Everyone is welcome to the
source code, but if you enhance it and distribute it, you must distribute
the source code for the enhancements. 
\item [{Keep~it~simple}] That is, use as few mathematical functions as
is necessary to achieve the desired objectives. To this day, ModAEM
uses only relatively simple first--order and second--order elements.
These are reliable, efficient, predictable, and sufficient for a wide
variety of practical applications.
\item [{Keep~it~parallel}] Make sure that the original design goal of
a parallel analytic element code remains. The current code should
be easily compiled with the --parallel switch with compilers that
support SMP hardware. I wish I had one.
\item [{Keep~it~portable}] ModAEM does not use extensions to the Fortran
95 language. It is clean and standard throughout. It has been successfully
compiled on numerous hardware and software platforms.
\item [{Assume~there's~a~preprocessor}] Nowadays, nobody wants to use
a model that doesn't have a nice preprocessor. As a result, ModAEM
simplifies the I/O model as much as possible. Many difficult tasks
are expected to be performed by the preprocessor, for example, developing
the topology of aquifer sub--domains. This makes the code more robust,
but harder to use ``by hand''. I for one prefer using a good preprocessor.
\item [{Document~the~code}] ModAEM has always had detailed documentation
built into the source code.
\item [{Have~fun!}] For me, ModAEM has always been a pleasure and a great
intellectual challenge. If it isn't fun, it isn't worth working on
it!
\end{description}

\section{How to read this manual}

The following conventions are used for formatting in this manual:
\begin{description}
\item [{\texttt{\textmd{typewriter}}}] text is used for sample input files
\item [{\textsf{\textmd{sans-serif}}}] text is used for file names and
other related items
\item [{\textsf{bold~sans-seri}f}] text is used for ModAEM script file
directives
\item [{bracketed~expressions}] such as $[L]$ contain the dimensions
for input data. Some examples are $[-]$ for dimensionless quantities,
$[L]$ for length, $[L/T]$ for length per time (e.g. a hydraulic
conductivity of $100\, ft/d$).
\end{description}

\section{Conventions for numeric input\label{sec:numeric-values}}

ModAEM makes use of ``free--format'' input for all numeric entries.
Since the computational heart of ModAEM is based on complex numbers,
most coordinate information is entered as complex quantities. In this
manual, the type of the input value is provided in curly braces, e.g.
$\{int\}$ for an integer value. When directed to provide a numeric
value, use the following rules:
\begin{description}
\item [{integer~values~$\{int\}$}] Positive or negative integer values
are allowed. ModAEM uses the Fortran 90 \texttt{SELECTED\_INT\_KIND}
function to specify Chapter 1 About the Book the size of integer values
(the default is 4 bytes per value). On 64-bit hardware (or on 32-bit
hardware with compilers that support ``quad--precision'' values,
you may overload the parameter \texttt{ModAEM\_Integer} in \textsf{u\_constants.f95}
and rebuild ModAEM. Note that this has not been tested; please report
success or failure to Vic Kelson.\\
For integers, do not provide a decimal point. For negative numbers,
the $-$sign \emph{must} be immediately before the first digit of
the value.

\begin{description}
\item [{right:}] \texttt{3~~~~-13241}
\item [{wrong:}] \texttt{-1.234~~~~7E+07}
\end{description}
\item [{logical~values~$\{logical\}$}] Values that may be either true
or false. ModAEM uses the Fortran 90 \texttt{SELECTED\_LOGICAL\_KIND}
function to specify the size of logical values (the default is 4 bytes
per value). If desired, you may overload the parameter \texttt{ModAEM\_Logical}
in \textsf{u\_constants.f95} and rebuild ModAEM.\\
For logicals, the only legal values are \texttt{T} (true) or \texttt{F}
(false). The value may be entered in either uppercase or lowercase.

\begin{description}
\item [{right:}] \texttt{T~~~~F}
\item [{wrong:}] \texttt{yes~~~~0}
\end{description}
\item [{real~values~$\{real\}$}] Positive or negative floating--point
values are allowed. ModAEM uses the Fortran 90 \texttt{SELECTED\_REAL\_KIND}
technique to specify the size of floating--point values (the default
is 8-bytes per value). On 64-bit hardware (or on 32-bit hardware with
compilers that support ``quad--precision'' values, you may overload
the parameter \texttt{ModAEM\_Real} in \textsf{u\_constants.f95} and
rebuild ModAEM%
\footnote{Although in principle it would be possible to specify single precision,
it is discouraged. On current 32-bit hardware, there is no speed benefit
from the use of single precision floating--point arithmetic.%
}. Note that this has not been tested; please report success or failure
to Vic Kelson.\\
Floating--point numbers may or may not contain a decimal point. If
exponential notation is desired, use the characters \texttt{$E\pm XX$}
as a suffix, where \texttt{XX} is the exponent. No space can lie between
the $-$ sign and the first digit of precision or between the mantissa
and exponent.

\begin{description}
\item [{right:}] \texttt{1.2~~~~-3.1415926~~~~6.02e+23}
\item [{wrong:}] \texttt{-~1.2~~~~7.01~e+99}
\end{description}
\item [{complex~values~$\{complex\}$}] Pairs of  floating--point values,
surrounded by parentheses and separated by a comma are allowed, where
the first value is the real part and the sceond value is the imaginary
part. ModAEM uses the Fortran 90 \texttt{SELECTED\_REAL\_KIND} technique
to specify the size of floating--point values, including complex values
(the default is 8-bytes per value). On 64-bit hardware (or on 32-bit
hardware with compilers that support ``quad--precision'' values,
you may overload the parameter \texttt{ModAEM\_Real} in \textsf{u\_constants.f95}
and rebuild ModAEM. Note that this has not been tested; please report
success or failure to Vic Kelson.\\
Complex numbers may or may not contain a decimal point. If exponential
notation is desired, use the characters \texttt{$E\pm XX$} as a suffix,
where XX
 is the exponent. No space can lie between the $-$ sign and the first
digit of precision or between the mantissa and exponent.

\begin{description}
\item [{right:}] \texttt{(1.2,3.45)~~~~(-3.1415926,0)}
\item [{wrong:}] \texttt{3+4i~~~~(3.54)}
\end{description}
\end{description}

\section{The right--hand rule\label{sec:The-right--hand-rule}}

For some elements, e.g the boundary segments defined by the \textsf{bdy}
directive in module AQU, the orientation of the points making up the
element is significant. In all cases, ModAEM makes use of the ``right--hand
rule''. The element is oriented such that if the modeler were to
stand at the first vertex facing the second vertex, the boundary condition
is specified on the ``left'' side of the element (the index finger
of the right hand, extended along the segment, points ``in''). For
example, for a flux--specified bdy element, the flux is numerically
positive if water moves from the right to the left. Similarly, a head--specified
boundary condition is to be met just to the left of the element.
