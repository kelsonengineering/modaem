\chapter{ModAEM script files\label{cha:script-files} }

ModAEM execution is controlled by the use of a ModAEM
script file (with the extension \textsf{.aem}), and
a ModAEM name file (called \textsf{modaem.nam}).
These files are the only input files that are required by the model.


\section{The ModAEM name file \textsf{modaem.nam\label{sub:name-file} }}

The standard library for the Fortran-95 language does not provide
a mechanism for gathering command-line arguments (e.g. the '\texttt{int
main(int argc, char {*}{*}argv)}' in a C or C++ program or \texttt{sys.argv{[}{]}}
in Python). Although nearly all current Fortran-95 compilers provide
a library routine for this task, they are not syntactically consistent.
One of the design objectives of the ModAEM project is that the code
should be as portable as possible, so language extensions have been
carefully avoided. Therefore, the official
ModAEM release code uses a file called \textsf{modaem.nam} in the
current working directory when the program begins execution to find
the ModAEM script file. The name file provides the base file name
for the ModAEM script file (and may provide other features in the
future). Developers are encouraged to add platform-specific support
for command-line arguments if they desire \footnote{It is expected that developers (particularly those who need a platform-specific
version of ModAEM) will add the ability to fetch command-line arguments
in the version of ModAEM that ships with their code. As of Fortran 95, the addition of this feature is compiler-dependent; however the Fortran 2003
library offers this capability, so future versions of ModAEM will likely take advantage of it. 

It is understood that some vendors may wish to add a platform-specific
GUI-style display to ModAEM (e.g. in a style similar to the MODFLOW/Win32
code that ships with the popular \emph{Groundwater Vistas} MODFLOW
GUI). It will be much appreciated if someone will make such a version
available under GPL. Although it is much preferred that such an extension
is released under GPL, the copyright holders understand the developer's
concerns and will consider requests (we make no guarantees \emph{a
priori}) for special licensing arrangements for such extensions.However,
we do not anticipate a willingness to engage in special licensing
exceptions for computational features that may be added to the code.%
}.

\subsection{Contents of the name file }

The ModAEM name file \textsf{modaem.nam} can be created with any text
editor. \textsf{modaem.nam} contains up to three lines of text, as
follows:

\paragraph{Line 1 -- Base file name for the model run}

The first line of the name file contains the ``base'' name of the
files for the model run. For example, if the model input data are
contained in a file called \textsf{modaem.aem}, then the contents
of \textsf{modaem.nam} would be:
\begin{verbatim}
modaem
\end{verbatim}
The extension \textsf{.aem} is appended to the file name by ModAEM.
In addition, two output files will be created using the same base
name, '\textsf{modaem}':
\begin{description}
\item [{\textsf{\textmd{modaem.err}}}] Will be written as the message
file. This file echoes program input and messages issued during execution.
It may be used as a run log file (and to see the results of some processing
directives). Although some model results can be sent to the message
file, it is not appropriate for extracting model results (e.g. heads)
for use in GUI programs; the inquiry
files written by module INQ (Section 5.9) are designed for this feature.
\item [{\textsf{\textmd{modaem.out}}}] Will be written as the output
file. This file recieves an HTML document listing of the solution
results, which may be useful in debugging. The output file is not
appropriate for loading results to be displayed in GUI programs; the
inquiry files written by module INQ
(Section 5.9) are designed for this feature.
\end{description}

\paragraph{Line 2 -- Name of the previous solution file}

The second file named in the name file is an optional file that contains
the saved results from a previous solution that are to be reloaded,
for example, to trace particles from a prior solution without the
need for re-solving. If it is provided, the results are read from
the previous solution file after the problem is defined (see the \textsf{\textbf{aem}}
directive and related information below), when the \textsf{\textbf{sol}}
directive is encountered. To ``solve'' the model byCool pictures?
Why would we do all this work if we weren't going to make cool pictures?
-- Mark Bakker

simply reloading the solution, issue the directive \texttt{sol~0}
after the problem definition is complete. If the previous solution
file is missing or if the file is empty, a warning will be issued
and execution will continue. However, if the file is not empty, a
fatal error results if the contents of the previous solution file
are inconsistent with the problem definition.

The file name may have any extension that the modeler desires, although
by convention, the extention \textsf{.pre} is recommended. If the
name of previous solution file is omitted from the name file, no previous
solution will be loaded.


\paragraph{Line 3 -- Name of the file where solution results are to be written}

The third file named in the name file is an optional file where the
results of the solution are to be written, for example, for use as
the initial condition in a future ModAEM run. If it is provided, ModAEM
will write the solution when the solution procedure is complete. The
file may be reloaded by specifying it in the second line of a future
name file (see above). 

The file name may have any extension that the modeler desires, although
by convention, the extention \textsf{.pre} is recommended. If the
name of the save file is omitted from the name file, no solution will
be saved.


\section{The AEM script file (\textsf{.}\textsf{\textmd{aem}} file) }

The AEM script file provides model elements and processing directives
to ModAEM. The AEM script file can have any base file name (as specified
in the modaem.nam file) and must have the extension \textsf{.aem}.
The AEM script file is a flat text file that can be created with any
text editor. Program directives are entered one per line.

The script file is divided into two sections, the problem
definition section (or AEM section)
and the processing section. Typically,
ModAEM script files look like this:
\begin{verbatim}
aem
  aqu
  ...
  # aquifer description goes here
  end
  # other module sections go here...
  wl0 10
    ... discharge-specified well data goes in here ...
    # end of well data 
  end 
  # end of aem data 
end
# processing directives go here ...
# End-of-data mark 
eod
\end{verbatim}

The attractive indentation style of the input file is optional, but highly recommended. By convention, each sub-section of the input file, e.g. \textsf{wl0} or \textsf{aqu}, 
is indented two spaces for readability. ModAEM ignores the indentation
when reading program input. Currently (at this writing, we are at
version 1.8.7), ModAEM behaves unpredictibly when it encounters
tab characters (ASCII 0x09) in input. The \textsf{aem} section of the input file (the portion contained between the \texttt{aem} and \texttt{end}
directives in lines 1 and 14 of the above listing) defines the actual elements that make up the model. Within the AEM
section, input for the various element definition modules (see Chapter
\ref{cha:script-files}) is provided.

The processing section follows the problem definition section of the
script file. The various processing directives that are available
are discussed in Chapter \ref{cha:Processing-Directives}.

\subsection{Directives which are common to all input modules}

The following directives are available in all ModAEM input modules.

\subsubsection{Comments}

Comment lines in the AEM script file start with a hash mark (\texttt{\#})
in the first column. Comment lines are ignored by ModaEM. For example:
\begin{verbatim}
# This is a comment line
\end{verbatim}

\subsubsection{Exiting a module (\textsf{\textmd{end}} directive) }

The \texttt{end} directive causes ModAEM to leave the current module.
For example, when in the \textsf{wl0} module (which is started with
the \texttt{wl0} directive), the end directive returns processing
to the AEM input module:
\begin{verbatim}
# the aem section is used to define the problem 
aem
   # other module sections go here...
  wl0 10
    ... discharge-specified well data goes in here ...
     # end of well data 
  end
  # end of aem data
end 
# processing directives go here ...
# End of data mark 
eod
\end{verbatim}

\subsection{Enabling debugging code (\textsf{\textsf{dbg}} directive) }

The \textsf{dbg} directive is used to turn code marked as 'debug'
code on or off during execution (useful for program debugging). Debug
code is enabled or disabled at the level of a specific module. The
ModAEM source code contains many assertions that can be used to test
for internal errors in the code. The \textsf{dbg} command does not
affect the detection of errors in program input, however. This command
will typically be used only by developers.

\subsection{Other directives for specific tasks }

\subsubsection{Begin defining a model problem domain (\textsf{aem} directive)}

The \textsf{aem} directive begins the problem definition section of the ModAEM
input file. See for a description of the various directives that may
be used in the problem definition section.
\paragraph{Usage:}
\begin{verbatim}
aem 
  ... put model definition directives here ... 
end
\end{verbatim}

\subsubsection{Processing directives}

The various processing directives that are available once a problem
has been defined in the AEM section are described in Chapter \ref{cha:Processing-Directives}.

