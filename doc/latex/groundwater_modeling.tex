\chapter{Groundwater modeling with ModAEM\label{cha:modeling-with-modaem}}

How do I build an analytic element model with ModAEM? How does ModAEM
work? 

Using ModAEM is much like using any other analytic element code. Somehow,
the modeler constructs a script file that controls the creation of
the model, first defining the aquifer properties and their distribution,
then adds elements to the aquifer that simulate various features in
the flow system, such as rivers and wells. The script file also directs
the solution of the model problem and uses various analytical tools
to extract results as grids, trace streamlines, and write reports.
Onc


\section{About the AEM}

This section introduces some important principles of analytic element
models, with the purpose of introducing the new ModAEM user to analytic
elements. Those who are interested in the ``gory details'' of analytic
elements should see Strack, 1989. For a more complete discussion of
modeling issues with analytic elements, please see Haitjema, 1995.


\subsubsection{What is an analytic element?}

The analytic element method is based on the superposition of analytic
functions. An \emph{analytic element} is a mathematical function that
may be superimposed with other analytic elements to create a complete
solution for a groundwater problem. In practice, two--dimensional,
steady--state analytic elements come in two varieties,
\begin{description}
\item [{Elements~that~satisfy~Laplace's~Equation}]~
\end{description}

\subsection{Discharge potentials }


\subsection{Example solutions}


\subsection{Analytic element functions and superposition}


\subsection{Boundary conditions for complex models}


\section{Using the AEM}


\subsection{Keep it simple}


\subsection{Stepwise modeling approach}


\subsection{Gotchas and troubleshooting}
