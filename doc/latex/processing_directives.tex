\chapter{Processing Directives\label{cha:Processing-Directives} }

This chapter describes processing directives that control the solution
process and general imspection of values for ModAEM. Many of the directives
described here are most commonly used for program debugging, experimentation
and development of example problems. When computing results that are
to be used for analysis purposes (e.g. within GUI--based modeling
environments), the modules GRI (grid/contour generation), EXT (data
extraction), INQ (element results inquiry), and OBS (observation data)%
\footnote{Available in ModAEM-1.4.1 and later versions.%
}. See Chapter \ref{cha:Analysis-modules} for details about the analysis
modules.

Each of these directives is given outside the ``AEM section'' of
the ModAEM script. The general layout of the script is as follows,
with the processing directives marked in bold face.

\begin{verbatim}
    aem
      aqu...
        ref...
        bdy...
        in0...
          ...
        end
      end
      # Other elements go here
    end
    \textbf{\#processingdirectivesgohere}
    \textbf{sol...}
    \textbf{hea...}
    eod
\end{verbatim}
In nearly every case, the first processing directive issued should
be \texttt{\textbf{sol x}}, where \texttt{\textbf{x}} is the number
of iterations (if a solution is to be loaded, use \texttt{\textbf{sol
0}} to simply load the saved results file). After the solution is
complete, the other directives will be useful.


\paragraph{Processing directives for solving and reporting solution results }
\begin{description}
\item [{\texttt{SOL}}] Solves the model, based on the available input and
results from a previous solution loaded from disk%
\footnote{Available in ModAEM-1.4.1 and later versions.%
}.
\item [{\texttt{RPT}}] Generates a report of all solution information in
HTML format.
\end{description}

\paragraph{Processing directives that retrieve data from the model at specific
points}
\begin{description}
\item [{\texttt{HEA}}] Reports the head at some location in the model to
the error log file.
\item [{\texttt{POT}}] Reports the complex potential at some location in
the model to the error log file.
\item [{\texttt{DIS}}] Reports the discharge at some location in the model
to the error log file.
\item [{\texttt{VEL}}] Reports the velocity at some location in the model
to the error log file.
\item [{\texttt{RCH}}] Reports the net exfiltration rate at some location
in the model to the error log file.
\item [{\texttt{THK}}] Reports the saturated thickness at some location
in the model to the error log file.
\end{description}

\paragraph{Processing directives that compute numerical approximations for model
testing}
\begin{description}
\item [{\texttt{GRA}}] Reports the numerical gradient at some location
in the model to the error log file.
\item [{\texttt{LAP}}] Reports the numerical laplacian at some location
in the model to the error log file.
\end{description}

\paragraph{Processing directives that compute values for a line segment}
\begin{description}
\item [{\texttt{FLO}}] Reports the integrated flow across a line segment
in the model to the error log file.
\end{description}
The remainder of this chapter describes the general--purpose processing
directives in detail.\newpage


\section{Directive SOL -- solve the model }

After a model has been defined (using the AEM module input section),
it must be solved prior to performing any analyses. ModAEM uses an
iterative solution scheme -- at each iteration, the solution is improved
based on the previous iteration, including the incorporation of non--linear
elements such as resistance line sinks or inhomogeneity boundaries
in regions where the flow is unconfined.

\paragraph{Usage:}
\begin{verbatim}
    sol niter relaxation
\end{verbatim}

\paragraph{Parameters for the sol directive:}
\begin{description}
  \item [{niter}] The number of iterations to be performed. $\{integer\}$\\
  The number of iterations to be used depends strongly on the problem
  to be solved. The following list describes the issues that determine
  the number of iterations needed for model convergence. Note that this
  list is a rule--of--thumb; the modeler should look closely at the solution
  to ensure that it has fully converged.
  \begin{description}
    \item [{1--2iterations}] Simple models that are linear at all points
      in the domain. Typically, this means that the flow is confined everywhere
      and that no resistance line sinks that can be removed from solution
      (that is, ``river'' and ``drain'' line sinks) are present. These
      problems should give accurate solutions with only 1--2 iterations.
    \item [{3--8iterations}] More complex models that make use of ``river''
      and ``drain'' line sinks, or have large areas in the model domain
      in which the flow is unconfined, but where the aquifer base elevation
      and thickness are constant everywhere will typically converge in 8
      iterations or less.
    \item [{moreiterations}] Very complex models in which the aquifer base
      elevation and/or thickness varies and the flow is unconfined, or problems
      where baseflow routing is in use. These problems may require 10 or more 
    i  terations to achieve convergence.
  \end{description}
  \item [{relaxation}] The relaxation factor to be used. $\{integer\}$\\
    This parameter instructs ModAEM to relax the solution during iterations.
    If $relaxation=1$, then ModAEM applies all of the computed adjustments
    in the strength coefficients on each iteration. For $0<relaxation<1$,
    the value of $relaxation$ is multiplied by the strenght adjustments.
    In some very complex models, this reduces the ``stiffness'' of the
    solution algorithm and may reduce the possibility of oscillations
    during iterations. In nearly all cases, this parameter should be set
    to 1.0.
\end{description}

\subsection{Loading a previous solution}

In ModAEM-1.8 and later versions, the modeler has the option of loading a 
previous solution. This is an advantage for large complex models; the previous
solution may be loaded prior to post-processing and analysis, e.g. particle
tracking or computing grids of heads. The problem definition, i.e.
the aquifer definition and creation of all elements must be complete
prior to loading the results. 

Because the previous solution is loaded in response to the \textsf{sol}
directive a special version of the directive, \texttt{\textbf{sol
0}}, is provided. By issuing the \textsf{sol} directive with no iterations,
the previous solution will be loaded, and all of ModAEM's internal
data structures will be restored, but no addition solution step will
be performed. Whenever a previous solution is to be used, the \texttt{\textbf{sol
0}} directive should be the first directive in the ModAEM script after
the AEM section is complete.

\subsection{Saving a solution for future re--use}

If the modeler provides a name for a ``solution save file'' on line
3 of the ModAEM name file, ModAEM stores the solution there as the
final step in the ``post--solve'' procedure. No directives are required
in the ModAEM script file.

\section{Directive RPT -- report the solution in HTML format}

After the solution is complete (see directive \textsf{SOL} above), 

\section{Directives that compute analytic values at a single point}

The following procesing directives compute a value and report it in
a human--readable format to the run--time message file.

\subsection{\label{HEA_directive}HEA - report the modeled head}
Reports the potentiometric head at a specified point to the message
file and to the console. A solution must be present (see the \texttt{sol}
directive) prior to issuing this directive. 


\paragraph{Usage:}
\begin{verbatim}
hea(x,y)
\end{verbatim}

\paragraph{Parameters for the \textsf{hea} directive }
\begin{description}
\item [{\texttt{(x,y)}}] The desired location in the complex plane $x+iy$.
$\{complex\}$
\end{description}

\paragraph{Example:}

The head at the location $(100,100)$ is reported in response to the
following directive:
\begin{verbatim}
HEA(100.0,100.0)
\end{verbatim}

\subsection{\label{POT_directive}POT - report the modeled complex potential}

Reports the complex potential at a specified point to the message
file and to the console. The complex potential is defined as $\Omega=\Phi+i\Psi$,
where $\Phi$ is the discharge potential and $\Psi$ is the streamfunction%
\footnote{Note that the streamfunction does not exist for problems that include
recharge; the reported streamfunction value is not useful in regions
where a recharge element (see module AS0 or PD0) is present.%
}. A solution must be present (see the \texttt{sol} directive) prior
to issuing this directive.


\paragraph{Usage:}
\begin{verbatim}
pot(x,y)
\end{verbatim}

\paragraph{Parameters for the \textsf{pot} directive }
\begin{description}
\item [{\texttt{(x,y)}}] The desired location in the complex plane $x+iy$.
$\{complex\}$
\end{description}

\paragraph{Example}

The complex potential at the location $(100,100)$ is reported in
response to the following directive:
\begin{verbatim}
pot(100.0,100.0)
\end{verbatim}

\subsection{\label{DIS_directive}DIS - report the total aquifer discharge }

Reports the total aquifer discharge $Q_{x}+iQ_{y}$ at a specified
point to the message file and to the console. The total discharge
is a two-dimensional analogue for the specific discharge:
\[
Q_{i}=Q_{x}+iQ_{y}=\int_{z_{bot}}^{z_{top}}(q_{x}+iq_{y})dz
\]
 where $q_{x}(z)$ and $q_{y}(z)$ are the horizontal components of
the specific discharge $q_{i}=-k\partial_{i}H$ at the elevation $z$,
and $z_{top}$ and $z_{bot}$ are the elevations of the aquifer top
and bottom, respectively%
\footnote{Since ModAEM is a 2--D code, it does not explicitly compute the integral.
The value is computed analytically by differentiating the discharge
potential $\Phi$ at the coordinate $x+iy$. See e.g. Strack (1989)
or Haitjema (1995) for details. A detailed description of the formulation
of ModAEM will be included in a future version of this book.%
}. A solution must be present (see the \texttt{sol} directive) prior
to issing this directive. 


\paragraph{Usage:}
\begin{verbatim}
dis(x,y)
\end{verbatim}

\paragraph{Parameters for the \textsf{dis} directive }
\begin{description}
\item [{\texttt{(x,y)}}] The desired location in the complex plane $x+iy$.
$\{complex\}$
\end{description}

\paragraph{Example}

The complex discharge at the location $(100,100)$ is reported in
response to the following directive:
\begin{verbatim}
dis(100.0,100.0)
\end{verbatim}

\subsection{\label{VEL_directive}VEL - report the horizontal groundwater velocity }

Reports the horizontal aquifer velocity $v_{x}+iv_{y}$ at a specified
point to the message file and to the console. The velocity should
be interpreted as the \emph{vertically--averaged} velocity at the
point in question:
\[
\bar{v}_{i}=\bar{v}_{x}+i\bar{v}_{y}=\frac{Q_{i}}{hn_{e}}
\]
 where $\bar{v}_{x}(z)$ and $\bar{v}_{y}$ are the reported horizontal
components of the average velocity, $h$ is the saturated thickness,
and $n_{e}$ is the effective porosity. A solution must be present
(see the \texttt{sol} directive) prior to issing this directive. 


\paragraph{Usage:}
\begin{verbatim}
vel(x,y)
\end{verbatim}

\paragraph{Parameters for the \textsf{vel} directive }
\begin{description}
\item [{\texttt{(x,y)}}] The desired location in the complex plane $x+iy$.
$\{complex\}$
\end{description}

\paragraph{Example}

The average groundwater velocity at the location $(100,100)$ is reported
in response to the following directive:
\begin{verbatim}
vel(100.0,100.0)
\end{verbatim}

\subsection{\label{RCH_directive}RCH - report the net recharge rate}

Reports the net rate of areal recharge at a specified point to the
message file and to the console. The reported recharge rate has a
sign consistent with the ModAEM conventions as described in e.g.,
Sections\ref{sec:as0-module} and \ref{sec:pd0-module}. The reported
rate is negative if the net recharge rate injects water into the aquifer,
positive if water is removed. A solution must be present (see the
\texttt{sol} directive) prior to issuing this directive.


\paragraph{Usage:}
\begin{verbatim}
rch(x,y)
\end{verbatim}

\paragraph{Parameters for the \textsf{rch} directive }
\begin{description}
\item [{\texttt{(x,y)}}] The desired location in the complex plane $x+iy$.
$\{complex\}$
\end{description}

\paragraph{Example}

The net recharge rate (negative for recharge) at the location $(100,100)$
is reported in response to the following directive:
\begin{verbatim}
rch(100.0,100.0)
\end{verbatim}

\section{Directives that compute numerical approximations for testing}

Two directives are provided that approximately compute the gradient
in the potential and laplacian of the potential, for the purpose of
testing the analytic functions that underlie all ModAEM computations.


\subsection{GRA\label{GRA_directive} - report the modeled numerical gradient
in potential}

Reports the \emph{numerical} gradient in the discharge potential)
at a specified point to the message
file and console. This directive is commonly used in program debugging;
the numerical gradient should have approximately the same value as
the total discharge (see the \texttt{dis} directive in Section \ref{DIS_directive}).
A solution must be present (see the \texttt{sol} directive) prior
to issing this directive. 

The aproximate gradient $(\hat{q}_{x},\hat{q}_{y})$ is computed at
$(x,y)$, a point in the 2D plane, using a spacing $\delta$ provided
by the user. The calculations are performed according to the following
expression.

\begin{eqnarray*}
\hat{q}_{x} & = & \frac{1}{\delta}\left[\Phi(x+\delta,y)-\Phi(x-\delta,y)\right]\\
\hat{q}_{y} & = & \frac{1}{\delta}\left[\Phi(x,y+\delta)-\Phi(x,y-\delta)\right]
\end{eqnarray*}
in general, it is expected that a more accurate estimate of the gradient
is obtained by using a small value of $\delta$.


\paragraph{Usage:}
\begin{verbatim}
gra{[}z{]}delta
\end{verbatim}
Reports the numerical gradient in the potential at the complex coordinate,
$z=(x,y)$, using the spacing $\delta$. Note that in Fortran free-format
reads, the two parts of the complex coordinate are provided as $(x,y)$
pairs. 


\paragraph{Example:}

The numerical gradient at the coordinate $(100,100)$, using a spacing
of $1.0$ is reported in response to the following directive:
\begin{verbatim}
gra(100.0,100.0)1.0
\end{verbatim}

\section{Directives that compute a net analytic value for a line segment}

ModAEM provides a directive that is used to report model results along
a line segment. Future versions of ModAEM may include additional directives
for line segments.


\subsection{FLO\label{cmd:flo} - report the total flow across a path}

Directs ModAEM to report the total integrated groundwater flow across
a linear path connecting two specified points to the message
file (.err file) and console. The value is computed as

\[
Q(z_{1},z_{2})=\int_{z_{1}}^{z_{2}}Q_{n}ds
\]
where
\begin{description}
\item [{$z_{1}$and$z_{2}$}] are
the points at the end-points of the line segment, $z_{j}=x_{j}+iy_{j}$
\item [{$Q_{n}$}] is the discharge normal to the line segment
$z_{1}z_{2}$
\item [{$s$}] is the direction along the line segment $z_{1}z_{2}$
\end{description}
A solution must be present (see the \texttt{sol} directive) prior
to issing this directive. 


\paragraph{Usage:}
\begin{verbatim}
flo(x1,y1)(x2,y2)
\end{verbatim}
Reports the integrated groundwater flux across the line-segment connecting
$(x_{1},y_{1})$ and $(x_{2,}y_{2})$ in units of $L^{3}/T$. The
sign of the result is determined by the right-hand rule (Section \ref{sec:The-right--hand-rule}).


\paragraph{Example:}

The total integrated flux across the line segment containing $(50,50)$
and $(100,100)$ is reported in response to the directive
\begin{verbatim}
flo(50.0,50.0)(100.0,100.0)
\end{verbatim}

\section{Extracting model results in machine-readable format (module INQ)\label{sec:module-INQ}}

Module INQ provides the ability to extract results from the model
in a machine-readable format that is convenient for post-processing
tools and graphical user interface (GUI) programs.

This section needs to be completed
